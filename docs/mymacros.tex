\newcommand{\trial}{\widetilde}
\newcommand{\subspace}{\widetilde}
\newcommand{\coeff}{\beta}
\newcommand{\coeffs}{\vec{\coeff}}
\newcommand{\coeffsopt}{\vec{\beta}_{\star}}

\newcommand{\verifyvalue}[1]{#1}
% \newcommand{\eg}{\textit{e.g.}\xspace}
% \newcommand{\ie}{\textit{i.e.}\xspace}
\newcommand{\eg}{\textit{e.g.}\xspace}
\newcommand{\ie}{\textit{i.e.}}

\newcommand{\ritzbasis}{X}
\newcommand{\ritzvec}{\tilde \psi}
\newcommand{\ritzval}{\tilde E}
\newcommand{\normmat}{\mathcal{N}}
\newcommand{\EC}{EC}
\newcommand{\NEC}{\ensuremath{N_{\rm EC}}}

\newcommand*{\diffdchar}{\mathrm{d}}    % or {ⅆ}, or {\mathrm{d}}, or whatever standard you’d like to adhere to
%\newcommand*{\dd}{\mathop{\diffdchar\!}}

\newcommand{\action}{\mathcal{S}}
\newcommand{\nbasis}{n_b}
\newcommand{\nsimulator}{N_h}
\newcommand{\nparam}{N_\theta}
\newcommand{\ntrain}{n_{\textup{train}}}

% \newcommand{\genkvp}{\mathcal{L}}
% \newcommand{\weights}{\beta}
% \newcommand{\kvpweights}{\vec{\weights}_\star}
\newcommand{\lagmult}{\lambda_\star}
\newcommand{\psitrial}{\widetilde \psi}
\newcommand{\dU}{\Delta \widetilde U}
\newcommand{\param}{\theta}
\newcommand{\params}{\boldsymbol{\theta}}
\newcommand{\umatrix}{\boldsymbol{u}}
\newcommand{\genkvp}{\mathcal{L}}

\newcommand{\identity}{\mathds{1}}
\newcommand{\trialfunc}{\xi}
\newcommand{\testfunc}{\zeta}

% \newcommand{\lecs}{\vec{a}}



\newcommand{\transpose}[1]{{#1}^{\intercal}}
\newcommand{\trans}{\intercal}

\newcommand{\etal}{\textit{et~al.}\xspace}

% Differential taken from Physics package
% Uses smart spacing for a nice look. Requires xparse.
% https://www.ctan.org/pkg/physics?lang=en

% First, the basics:
\def\diffd{\mathrm{d}}  % Upright differentials
% \def\diffd{d}  % Italic differentials

% Now add spacing:
% Derivatives
\DeclareDocumentCommand\differential{ o g d() }{ % Differential 'd'
    % o: optional n for nth differential
    % g: optional argument for readability and to control spacing
    % d: long-form as in d(cos x)
    \IfNoValueTF{#2}{
        \IfNoValueTF{#3}
            {\diffd\IfNoValueTF{#1}{}{^{#1}}}
            {\mathinner{\diffd\IfNoValueTF{#1}{}{^{#1}}\argopen(#3\argclose)}}
        }
        {\mathinner{\diffd\IfNoValueTF{#1}{}{^{#1}}#2} \IfNoValueTF{#3}{}{(#3)}}
    }
\DeclareDocumentCommand\dd{}{\differential} % Shorthand for \differential
